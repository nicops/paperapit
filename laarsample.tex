%\input{tcilatex}


\documentstyle[laarstyle]{article}
%%%%%%%%%%%%%%%%%%%%%%%%%%%%%%%%%%%%%%%%%%%%%%%%%%%%%%%%%%%%%%%%%%%%%%%%%%%%%%%%%%%%%%%%%%%%%%%%%%%%%%%%%%%%%%%%%%%%%%%%%%%%
%TCIDATA{Created=Mon Jun 26 12:16:44 2000}
%TCIDATA{LastRevised=Fri Feb 21 17:20:01 2003}
%TCIDATA{Language=American English}

\affiliation{
\dag Chemical Engng. Department, Caltech, Pasadena, CA 91125, USA \\
{\ xalpha@caltech.edu} \\
\ddag Dto. de Ing. Mec\'{a}nica, Univ. Nac. del Litoral, 7500 Santa Fe
, Argentina \\
{\ ybetha@intec.edu.ar}
}
%\input{tcilatex}
\begin{document}

\title{CAMERA-READY INSTRUCTIONS (LAAR)}
\author{X. X. ALPHA$^{\dag }$ and Y. Y. BETHA$^{\ddag }$}
\maketitle

\abstract
These instructions are presented to assist authors in preparing a typescript
which is suitable for direct photo-offset reproduction. The abstract should
not exceed 200, 150 and 50 words for Review Papers, Articles and Notes
respectively.\endabstract

%TCIMACRO{\TeXButton{keywords}{\keywords}}
%BeginExpansion
\keywords%
%EndExpansion
Up to five keywords should be provided.\endkeywords

\section{INTRODUCTION}

The authors are fully responsible for the printing quality of their papers,
and are kindly requested to observe carefully the following instructions for
the preparation of their manuscripts. This document is itself an example of
the desired layout for camera-ready papers.

There is no limit to the length of the paper. However, it is recommended
that Review Papers, Articles and Notes do not exceed 10, 8, and 4 pages,
respectively. The publisher will charge a mandatory page charge of US\$ 75
per page for each page in excess of 10, 8 or 4 depending on the type of
contributions.

\section{METHODS}

\subsection{Typing Instructions}

Manuscripts must be typed in a two column format within a box of 165 mm $%
\times $ 245 mm. Column width should be 80 mm with a 5 mm space between both
columns. Authors should send to the Editorial Board a high quality hard copy
(laser printers or letter quality printers should be used) and an electronic
version (in *.pdf or *.ps files) of their revised final version.

\subsection{Font Size and Spacing Lines}

Manuscripts must be typed with single line spacing. Increase line spacing
only when necessary for subscripts and/or superscripts. Font size in text
and any drawings should be, at least, of approximately 2 mm high. The
minimum font size for the body of the text is 10 point (1 point=0.35 mm).

\subsection{Illustrative Materials}

Illustrations must be originals or photographic prints of originals. Arrange
them throughout the paper and do not group them together at the end. Figure
captions must be typed beneath each figure.

Tables can be typed directly onto the sheets. Table headings should be as
brief as possible and typed directly above the table.

The words ``Figure'' and ``Equation'' should be shortened to ``Fig.'' and
``Eq.'' whenever they occur within a sentence. They should, however, be
written in full when they appear at the beginning of a sentence. Equations,
figures and tables must be numbered in Arabic numbers. The number of the
equation should be aligned to the right margin

\begin{equation}
\dot{x}=f_{1}\left( x_{1},t\right) +\alpha x_{2}\sin \left( \theta \right)
\label{eq1}
\end{equation}
and the equation itself should be centered.

\subsection{Format of References}

Within the text, the references should be quoted by author 's surname and
year. Use both authors' surname if two, first author and ``{\it et al.}''
when more than two. Depending on sense, either the year only is placed in
parentheses, or the author's surname and year separated by a comma are
placed together in parentheses as shown below. For example: ``Komatsue
(1977a) discovered...''. ``Recent results (Arnikar {\it et al.}, 1970;
Holland, 1981) indicate...''. ``Komatsue (1977a, b) found...''. ``Early
investigators (Komatsue and Simand, 1987) thought...''. The list of
references should be typed in {\it alphabetical order} and follow the
general style given below as an example.

\section{CONCLUSIONS}

It is important to adhere to these ``rules'' so that the volume can be
produced quickly, efficiently and in a fully appealing readable form.

\begin{thebibliography}{9}
\bibitem{arnikar1965}  Arnikar, H.J., T.S. Rao and A. Bodne, ``A gas
chromatographic study of the kinetics of the uncatalysed esterification of
aceticacid by ethanol,'' {\it J. Chromatog.}, {\bf 47}, 265-268 (1970).

\bibitem{Holland1980}  Holland, C.D., {\it Fundamentals of Multicomponent
Distillation}, McGraw-Hill, New York (1981).

\bibitem{Homatsue1977a}  Komatsue, L., ``Application of the relaxation
method for solving reacting distillation problems,'' {\it J. Chem. Eng. Japan%
}, {\bf 10}, 200-205 (1977a).

\bibitem{Komatsue1977b}  Komatsue, L., ``A new method of convergence for
solving reacting distillation problems,'' {\it J. Chem. Eng. Japan}, {\bf 10}%
, 292-297 (1977b).

\bibitem{Komatsue1987}  Komatsue, L. and J. Simand, ``Simulation of reactive
distillation by the inside-outside method,'' {\it Proc. 37th Chem. Eng. Conf.%
}, Montreal, Canada, 365-367 (1987).
\end{thebibliography}

\end{document}
