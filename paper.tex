%======================================================================
% DVCS paper!
%======================================================================
\documentclass[%
	%draft,
	%submission,
	%compressed,
	final,
	%
	%technote,
	%internal,
	%submitted,
	%inpress,
	%reprint,
	%
	%titlepage,
	notitlepage,
	%anonymous,
	narroweqnarray,
	inline,
	twoside,
        %invited,
	]{ieee}

\newcommand{\latexiie}{\LaTeX2{\Large$_\varepsilon$}}
\usepackage{graphicx}
%\usepackage{ieeetsp}	% if you want the "trans. sig. pro." style
%\usepackage{ieeetc}	% if you want the "trans. comp." style
%\usepackage{ieeeimtc}	% if you want the IMTC conference style

% Use the `endfloat' package to move figures and tables to the end
% of the paper. Useful for `submission' mode.
%\usepackage {endfloat}

% Use the `times' package to use Helvetica and Times-Roman fonts
% instead of the standard Computer Modern fonts. Useful for the 
% IEEE Computer Society transactions.
%\usepackage{times}
% (Note: If you have the commercial package `mathtime,' (from 
% y&y (http://www.yandy.com), it is much better, but the `times' 
% package works too). So, if you have it...
%\usepackage {mathtime}

% for any plug-in code... insert it here. For example, the CDC style...
%\usepackage{ieeecdc}

\begin{document}

%----------------------------------------------------------------------
% Title Information, Abstract and Keywords
%----------------------------------------------------------------------
\title[Distributed Version Control Systems]{%
       Distributed Version Control Systems}

% format author this way for journal articles.
% MAKE SURE THERE ARE NO SPACES BEFORE A \member OR \authorinfo
% COMMAND (this also means `don't break the line before these
% commands).
\author[GRUPOAPIT]{Demian Alonso,
\and{} Bruno Bonnano,
\and{} Nicol\'{a}s Calligaro
\and{}and Nicol\'{a}s Perez Santoro
}

\journal{Universidad Tecnol\'{o}gica Nacional, Facultad Regional Buenos Aires}
%\titletext{, VOL.\ 46, NO.\ 6, DECEMBER\ 1997}
%\ieeecopyright{0018--9456/97\$10.00 \copyright\ 1997 IEEE}
%\lognumber{xxxxxxx}
%\pubitemident{S 0018--9456(97)09426--6}
%\loginfo{Manuscript received September 27, 1997.}
\firstpage{1}

%\confplacedate{Buenos Aires, Argentina, April 18, 2010}

\maketitle               

\begin{abstract} 
Abstract to be done
\end{abstract}

\begin{keywords}
to be done
\end{keywords}

%----------------------------------------------------------------------
% SECTION I: Introduction
%----------------------------------------------------------------------
\section{Introduction}

%It is given that, in these times, modern software projects need a tool to 
%automatically track changes made in the source files, called Version Control Systems (VCS). 

\PARstart Version Control Systems (VCS) usually track the whole history of the
source code in a central server that everyone that can access. This 
central server contains all the changes and branches of the code and 
the users that can read the versions or commit changes have to connect 
to this repository. The most popular today example of this approach is SVN, which is a 
widely used VCS.

Distributed Version Control Systems (DVCS) are a modern way of managing
revisions of software that do not have a centralized server, for this reason 
they are also called De-centralized Version Control Systems. Everybody has a 
copy of the full repository with all changes made in the history.

The concept of distributed repositories came up first on the commercial product 
Sun TeamWare (\emph{good citation needed}), designed by Larry McVoy who later went 
on to design BitKeeper, another commercial VCS which expanded on those ideas. 
BitKeeper was used from 2002 to 2005 to manage the Linux kernel source until 
the license to use BitKeeper was finished. 
To replace the use of BitKeeper in the Linux kernel, two projects started, Git (by
Linus Torvalds \cite{linusgit} and 
Mercurial, both now mature DVCS. Other open source DVCS in use today are Bazaar, 
developed by Canonical Ltd. and used to maintain the Ubuntu codebase, and Darcs, based on 
the idea of tracking only patches and not versions.

%Vamos a contar cuando surgió esta tecnología y sus principales 
%exponentes en la actualidad. Vamos a basarnos en GIT pero hay muchas 
%similitudes con otros sistemas de versionado distribuidos como Mercurial. 
%Un poco se contará las diferencias fundamentales entre las herramientas de 
%versionado centralizadas y distribuidas, simplemente nombrando las diferencias 
%pero no explicándolas. 

We're going to do a comparison between centralized and decentralized control version systems,
highlighting the differences and advantages and disadvantages of both approachs in various 
contexts, both on a conceptual and practical level. In this paper we'll base our explanations 
on Git but there are no big differences between 
most of these DVCS, especially between Git and Mercurial that are largely similar, with 
most commands having the same names in both tools.

%La idea del paper sería contar las ventajas que puede 
%tener este esquema dentro de un ambiente enterprise de código cerrado, en contraste 
%a un ambiente open source donde el esquema distribuido tiene incluso más sentido.

%Ejemplo de cita \cite{linusgit} 


%----------------------------------------------------------------------
% SECTION II: Architecture
%----------------------------------------------------------------------
\section{Architecture}

A DVCS has many features that makes it perfect for distributed and non-distributed developments. Almost every feature is based on the fact that DVCS is distributed instead of centralized and therefore if you work distributed then merges would happen on a regular basis and they should be as fast and easy as possible, providing developers with a whole new set of tools and relieve them almost completely from having to code according to the rules required to work with a centralized CVS like system. 

\subsection{Distributed}
The idea behind being distributed means that no server has the full content of a repository but rather a clone of it with the changes that a particular developer has commited. This allows users to have their own clone on their own computer and work in a offline environment while being able to version their changes. It also makes the whole versioning system faster.

\subsection{Change-sets}
You could also create several other clones for every feature or bug that they want to develop and push those changes to another clone, a peer's repository for instance, and pull some changes from another peer to your local clone. To able to do this, a DVCS tracks change-sets instead of changes in individual files so that it can keep track of all the changes needed for a feature/bug to work. 

\subsection{Patchs}
The idea of having several clones, one for each feature or bug might be problematic so instead of doing that, you could have all your changes on a single clone and then select which one of those changes you want to commit. You can even select which parts of a single file you want to commit and which ones you don't, then group those changes and commit them in a very atomic way.

\subsection{Flexibility}
If you need a centralized repository you could still use a DVCS and all it's features while pushing all the changes to a particular repository. This shows how flexible a DVCS is, to the point of being able to describe it as a super-set of a centralized CVS although in really it is not.

\subsection{Trust Based Security}
Security on DVCS is based on the "circle of trust" model which implies that you trust only on a very specific group of people and you only pull from those people which in turn trust in their own group of people and only pull from that other group of people. This creates a chain of trust so that even if you don't trust some one directly by scalating their changes you could end up having his code only after it has been reviewed by the people you really trust.

\subsection{Small, Coherent and Logical Commits}
Due to the fact that you can select very specific pieces of your code to build a commit you can make commits very small and atomic making the merging process more easy by selecting only those commits that won't break your code. This technique of only selecting the changes that you want is called ``cherry-picking'' and it has the advantage of using the revision tree as a 
series of \emph{changesets} or \emph{patches} instead of just a series of static versions. The distributed model enhances this by allowing to make commits whenever you want to your local copy without breaking anyone else's code. Many little commits can be grouped as a bigger, conceptual commit in the case that you need those changes to occur at the same time and not individually. 

\subsection{Tagging and Branching}
Most DVCS allows the user to attach an arbitrary description the a tag so you can store the whole release announcements along with the tag of that version. Moreover, as with the commit, the identity of the user who made the tag is stored and can be securely verified by a cryptographical PGP signature that is embedded on the tag

\subsection{Merging}
The merging process has two possible scenarios:
\emph{Fast-forward merge:} The changes has been applied to only one of the branches and therefore those changes are replayed to the other branch automatically.
\emph{Three-way merge:} When changes exists on both branches, the DVCS tool will try to automatically merge them and if any conflict arises then it reports it to you and let you resolve them. 
The merging process\cite{svnmerging} between branches preserves all the history of 
both branches on many DVCS tools.

\subsection{Pushing and pulling}
When we want to share our work with the rest of the team we have two basic commands, 
\emph{push} and \emph{pull}.

\subsection{Pushing}
The push process refers to taking all the changes previously selected and adding them to the repository of another peer. In practice we usually don't do this because we shouldn't push changes to a peer's working copy of the repository so the push process is usually done agains a ``central repository'' where all the team make their pushes too. This logic is similar to centralized VCS, but we must keep in mind that we could have a lot of ``central repositories'' for different parts of our project or even for different parts of the team. Testing could have its own central repository, while development has another copy of it and changes could be pushed among them very easily.

\subsection{Pulling}
Pull is the opposite action of push and it is the combination of two other commands, ``fetch'' and ``merge''. The fetch command takes the changes from a peer's repository and copies them to the local repository. After that, the merge process combines the changes you've just downloaded with the ones on your local repository. You can do those two actions separately or altogether by using the ``pull'' command.

\section{Comparison}

\subsection{Commits}
In a DVCS the nature of the commits is different to centralized VCS, since commits are made in the local repository only. 
Commiting in centralized VCS means ``to share the change with everyone that queries the centralized server'' while in a 
DVCS means to store a change in the history, which is not shared until it's pushed or pulled into another repository.

Also, DVCS allow to apply a technique called ``cherry-picking'', 
which consists of ``picking'' what changes you want to use in a version, and build 
a new version containing only those changes. This way the revision tree is seen as a 
series of \emph{changesets} or \emph{patches} instead of just a series of static versions.

%Estos dos conceptos son distintos al check in / check out centralizado, 
%consisten en el envío o recepción de commits entre dos repositorios. Aquí 
%explicaremos un poco en que consisten y cuando hay que usar uno u otro.

%----------------------------------------------------------------------
% SECTION III: Should I use it in my project?
%----------------------------------------------------------------------
%Explicar los casos donde resulta interesante aplicar esta herramienta. 
%También la idea es mostrar configuraciones posibles para la empresa donde 
%se vea la forma de hacer cosas que con un sistema no distribuido no sería 
%posible, o como es posible hacer lo mismo que se podía hacer con svn 
%(usando un repositorio que tenga el branch principal), de manera que 
%no te limita.

\section{Possible workflows}

As mentioned before, this tool allows for differents workflows that just weren't 
possible before.

Changes can be just shared between peers, without needing to access the central 
repository. Branches can be created instantly, and merging them is easier since 
the whole history of changesets is preserved instead of just taking the diff of 
the versions.

Push and pull are much more versatile than checking in and out, since you can push 
changes into many repositories and pull from many others, not just a central one.
In fact, there is no need to push at all:
everyone could be just pulling changes from each other. For example, two co-workers could work 
in a feature, pulling from each other, then let a third one fetch them and do a code review, and then 
discard it or merge it. A fourth could fetch all reviewed changes into a test version, to do an 
acceptance test, and then merge the changes into a final version.

\subsection{Peer code review}
If doing code review is a common practice, then using a DVCS can aid to this process. Since changes can be shared
among any workstation, the reviewer can fetch the changes from the reviewee's workstation. The reviewer might be
working on some other feature and may have it's workspace in an unstable state. But this is not a problem since the
reviewer can just clone the reviewee's repository and can have it sepparatedly from its current changes.
This also prevented them from sharing the code through a public repository, which can present some disadvantages like
a third developer using code that is not yet intended for general use.

After the review, they can push changes to some public repository for everyone to use, or the reviewer can push changes to the reviewee repository for further fixes. All of this is completely isolated from the rest of the team and with no commit history loose.

\subsection{Independent subteams}
If the application being developed requires a large amount of people working at the same time, it is very likely that they are working on different subsystems. Each subteam can be working on a different repository, and so the chances of one team messing with the other teams build is completely removed. This helps each team to mantain their own build and if a problem on the build arises, then it is something to be solved between a reduced group of people.

Then both repositories can be merged togheter into an integration one by pulling from each of them. If there are conflicts which cannot be resolved easily, it is possible to ask the team who caused the conflict to solve it since all history is preserved. Then this team can clone the integration repository and merge it with its working repository. They can ask the person in charge to pull the changes back again.

\subsection{Centralized repository with integration manager}
It is possible to have a single central repository where all developers have read access. Everyone has an official public version from which they can work from. Then, to add new change sets to the official version, an integration manager can pull the changes from each developer (or each team) to have all the changes tested and reviewed. When the build is good enough the integration manager can push the changes to the central repository so all developers have the new version available.
It is also possible to have an automatic integration manager. This server will accept pushes, run all tests and if everything is ok it will then push this changes to the central repository. Otherwise it can discard the changes and signal the developer the reasons (most likely through an email).


%----------------------------------------------------------------------
% SECTION IV: Then, why should I use it in my project?
%----------------------------------------------------------------------
%Si mi proyecto es apto para aplicar la herramienta por qué debería utilizarla en 
%vez de utilizar CVS/SVN/otros, mostrando los costos y ventajas y desventajas,
% siempre desde el punto de vista "closed source" (debatible).
\section{Then, why should I use it in my project?}
Does everybody need these features? Speed, offline working, 
and easyness of branching/merging are nice features even if you work simply pulling 
and pushing changes against a central repository and not every peer out there. In some 
projects and with some tools, the advantage of going distributed may not be clear. 
For starters, in many closed source projects there is no need nor desire to have anything
other than centralized repositories, and while you can use a DVCS with a centralized
repository, there may not be enough to gain to balance the cost of changing. Distributed 
versioning \emph{is} more complex, and there is a ``conceptual overhead'' in the usage 
of it, even if most of the time you can ignore most features and use just what you need 
(even for one person proyects, Git and Mercurial may be easier to use than SVN).
Centralized VCS also have their advantages, you don't need to have the whole history in your 
personal repository so you can save space, for example in Perforce you can check out only a part of the 
repository\cite{perforceclientspec} while in Git repositories can't handle millions of file 
without getting a performance hit on some operations\cite{linusgit}.

Another issue with these systems is that the tool support and various plugins just isn't as 
mature as systems like SVN, which are much more mature in that regard. Git, in particular, is 
biased on Linux and doesn't really work out of the box in Windows, but works with Cygwin. 
Mercurial is much more portable, but some tools are also incomplete, like hg-subversion (a tool 
to interoperate with svn). Git has git-svn which allows to push and pull from svn repositories, 
easing transitions.


%----------------------------------------------------------------------
% SECTION VII: Conclusions
%----------------------------------------------------------------------
\section{Conclusion}

\emph{to be done}

%----------------------------------------------------------------------
\section{Glossary}
{\renewcommand{\labelitemi}{$\triangleright$}
\begin{itemize}
\item CVS
\item Centralized CVS
\item Distributed/de-centralized CVS
\end{itemize}

\begin{thebibliography}{1}

% \bibitem{lamport}
% Leslie Lamport,
% \newblock {\em A Document Preparation System: {\LaTeX} User's Guide and
%   Reference Manual},
% \newblock Addison-Wesley, Reading, MA, 2nd edition, 1994.
% \newblock Be sure to get the updated version for \latexiie!
% 
% \bibitem{goossens}
% Michel Goossens, Frank Mittelbach, and Alexander Samarin,
% \newblock {\em The {\LaTeX} Companion},
% \newblock Addison-Wesley, Reading, MA, 1994.

\bibitem{linusgit}
Linus Torvalds
\newblock \emph{Google Talk presentation on Git}

\bibitem{perforceclientspec}
Perforce documentation on ``Editing Client Specs''

\newblock \emph{
http://www.perforce.com/perforce/doc.current/manuals/p4web
/help/editclient.html
}

\bibitem{svnmerging}
\newblock \emph{ Git-SVN Crash Course}

\newblock \emph{http://git.or.cz/course/svn.html#merge}
\end{thebibliography}

%----------------------------------------------------------------------

\end{document}
