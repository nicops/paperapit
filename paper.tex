%======================================================================
% DVCS paper!
%======================================================================
\documentclass[%
	%draft,
	%submission,
	%compressed,
	final,
	%
	%technote,
	%internal,
	%submitted,
	%inpress,
	%reprint,
	%
	%titlepage,
	notitlepage,
	%anonymous,
	narroweqnarray,
	inline,
	twoside,
        %invited,
	]{ieee}

\newcommand{\latexiie}{\LaTeX2{\Large$_\varepsilon$}}

%\usepackage{ieeetsp}	% if you want the "trans. sig. pro." style
%\usepackage{ieeetc}	% if you want the "trans. comp." style
%\usepackage{ieeeimtc}	% if you want the IMTC conference style

% Use the `endfloat' package to move figures and tables to the end
% of the paper. Useful for `submission' mode.
%\usepackage {endfloat}

% Use the `times' package to use Helvetica and Times-Roman fonts
% instead of the standard Computer Modern fonts. Useful for the 
% IEEE Computer Society transactions.
%\usepackage{times}
% (Note: If you have the commercial package `mathtime,' (from 
% y&y (http://www.yandy.com), it is much better, but the `times' 
% package works too). So, if you have it...
%\usepackage {mathtime}

% for any plug-in code... insert it here. For example, the CDC style...
%\usepackage{ieeecdc}

\begin{document}

%----------------------------------------------------------------------
% Title Information, Abstract and Keywords
%----------------------------------------------------------------------
\title[Investigation on GIT]{%
       Investigation on GIT}

% format author this way for journal articles.
% MAKE SURE THERE ARE NO SPACES BEFORE A \member OR \authorinfo
% COMMAND (this also means `don't break the line before these
% commands).
\author[GRUPOAPIT]{Damian Alonso,%
\and{} Bruno Bonnano
\and{} Nicol\'{a}s Calligaro
\and{}and Nicol\'{a}s Perez Santoro
}

\journal{Universidad Tecnol\'{o}gica Nacional, Facultad Regional Buenos Aires}
%\titletext{, VOL.\ 46, NO.\ 6, DECEMBER\ 1997}
%\ieeecopyright{0018--9456/97\$10.00 \copyright\ 1997 IEEE}
%\lognumber{xxxxxxx}
%\pubitemident{S 0018--9456(97)09426--6}
%\loginfo{Manuscript received September 27, 1997.}
\firstpage{1}

%\confplacedate{Buenos Aires, Argentina, April 18, 2010}

\maketitle               

\begin{abstract} 
Abstract to be done
\end{abstract}

\begin{keywords}
to be done
\end{keywords}

%----------------------------------------------------------------------
% SECTION I: Introduction
%----------------------------------------------------------------------
\section{Introduction}

%It is given that, in these times, modern software projects need a tool to 
%automatically track changes made in the source files, called Version Control Systems (VCS). 

\PARstart Version Control Systems (VCS) usually track the whole history of the
source code in a central server that everyone that can access. This 
central server contains all the changes and branches of the code and 
the users that can read the versions or commit changes have to connect 
to this repository. The most popular today example of this approach is SVN, which is a 
widely used VCS.

Distributed Version Control Systems (DVCS) are a modern way of managing
revisions of software that do not have a centralized server, for this reason 
they are also called De-centralized Version Control Systems. Everybody has a 
copy of the full repository with all changes made in the history.

The concept of distributed repositories came up first on the commercial product 
Sun TeamWare (\emph{good citation needed}), designed by Larry McVoy who later went 
on to design BitKeeper, another commercial VCS which expanded on those ideas. 
BitKeeper was used from 2002 to 2005 to manage the Linux kernel source until 
the license to use BitKeeper was finished. 
To replace the use of BitKeeper in the Linux kernel, two projects started, Git and 
Mercurial, both now mature DVCS. Other open source DVCS in use today are Bazaar, 
developed by Canonical Ltd. and used to maintain the Ubuntu codebase, and Darcs, based on 
the idea of tracking only patches and not versions.

%Vamos a contar cuando surgió esta tecnología y sus principales 
%exponentes en la actualidad. Vamos a basarnos en GIT pero hay muchas 
%similitudes con otros sistemas de versionado distribuidos como Mercurial. 
%Un poco se contará las diferencias fundamentales entre las herramientas de 
%versionado centralizadas y distribuidas, simplemente nombrando las diferencias 
%pero no explicándolas. 

We're going to do a comparison between centralized and decentralized control version systems,
highlighting the differences and advantages and disadvantages of both approachs in various 
contexts, both on a conceptual and practical level. In this paper we'll base our explanations 
on Git but there are no big differences between 
most of these DVCS, especially between Git and Mercurial that are largely similar, with 
most commands having the same names in both tools.

%La idea del paper sería contar las ventajas que puede 
%tener este esquema dentro de un ambiente enterprise de código cerrado, en contraste 
%a un ambiente open source donde el esquema distribuido tiene incluso más sentido.

Ejemplo de cita \cite{linusgit} 


%----------------------------------------------------------------------
% SECTION II: Architecture
%----------------------------------------------------------------------
\section{Architecture}

In a DVCS like Git, everybody has the full repository in their hard drives.
Since there is no central server, operations on the repository all are done 
offline. This means that actions like commiting do not depend on having a 
network connection, since commits are made on your repository.

Changes can thus be shared between everyone passing changes between repositories, 
without the need of making those changes in a central repository.

% Explicaremos el principio de funcionamiento de la herramienta distribuida 
% y sus principales objetivos, contrastando el flujo de trabajo con el de un 
% sistema de control de versiones centralizado. Básicamente que cada uno tiene 
% su propio repositorio, que las versiones pueden ser infinitas (por más que 
% haya una(s) maestra(s)), que se puede trabajar offline, etc.

% -habria que agregar una imagen de muetsra

\subsection{Atomic Commits}

In a centralized VCS, most of the time commits aren't done until you are sure that 
the change is well written and can be actually shared to everybody else, since the 
commit is going to be shared as soon as it's done. In a DVCS the nature of the commits 
change, since commits are made in the local repository, they tend to be more atomic, 
and commits aren't huge changesets but can be logical, coherent changes.
To do this in a centralized VCS, one would have to create a new branch of the source code, 
and then merge the branch, but creating and merging branches all the time is 
resource-consuming, and branches are seen by everybody in the 
%En el sistema distribuido, hay una diferencia conceptual con los sistemas 
%centralizados acerca de la naturaleza de un commit. Se tiende a hacer commits
%más atómicos y lógicos ya que el commit se hace en el repositorio propio, 
%entonces no hay preocupaciones como romper el build.

\subsection{Branching and Merging}

En un sistema distribuido, tener muchísimos branches es sencillo, de hecho 
es más que sencillo, es "natural". Lo que permite un sistema distribuido es 
no solo tener branches baratos sino también simplificar los merges. 

\subsection{Pushing and fetching/pulling}

Estos dos conceptos son distintos al check in / check out centralizado, 
consisten en el envío o recepción de commits entre dos repositorios. Aquí 
explicaremos un poco en que consisten y cuando hay que usar uno u otro.

%----------------------------------------------------------------------
% SECTION III: Should I use it in my project?
%----------------------------------------------------------------------
\section{Should I use it in my project?}

Explicar los casos donde resulta interesante aplicar esta herramienta. 
También la idea es mostrar configuraciones posibles para la empresa donde 
se vea la forma de hacer cosas que con un sistema no distribuido no sería 
posible, o como es posible hacer lo mismo que se podía hacer con svn 
(usando un repositorio que tenga el branch principal), de manera que 
no te limita.

%----------------------------------------------------------------------
% SECTION IV: Then, why should I use it in my project?
%----------------------------------------------------------------------
\section{Then, why should I use it in my project?}

Explicar los casos donde resulta interesante aplicar esta herramienta. 
También la idea es mostrar configuraciones posibles para la empresa donde 
se vea la forma de hacer cosas que con un sistema no distribuido no sería 
posible, o como es posible hacer lo mismo que se podía hacer con svn 
(usando un repositorio que tenga el branch principal), de manera que 
no te limita.

%----------------------------------------------------------------------
% SECTION VII: Conclusions
%----------------------------------------------------------------------
\section{Conclusion}

\emph{to be done}

%----------------------------------------------------------------------


\begin{thebibliography}{1}

% \bibitem{lamport}
% Leslie Lamport,
% \newblock {\em A Document Preparation System: {\LaTeX} User's Guide and
%   Reference Manual},
% \newblock Addison-Wesley, Reading, MA, 2nd edition, 1994.
% \newblock Be sure to get the updated version for \latexiie!
% 
% \bibitem{goossens}
% Michel Goossens, Frank Mittelbach, and Alexander Samarin,
% \newblock {\em The {\LaTeX} Companion},
% \newblock Addison-Wesley, Reading, MA, 1994.

\bibitem{linusgit}
Linus Torvalds
\newblock Google Talk presentation on Git

\end{thebibliography}

%----------------------------------------------------------------------

\end{document}
